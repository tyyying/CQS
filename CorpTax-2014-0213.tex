\documentclass[12pt]{article}
%%%%%%%%%%%%%%%%%%%%%%%%%%%%%%%%%%%%%%%%%%%%%%%%%%%%%%%%%%%%%%%%%%%%%%%%%%%%%%%%%%%%%%%%%%%%%%%%%%%%%%%%%%%%%%%%%%%%%%%%%%%%
\usepackage{amsfonts}
\usepackage{graphicx}
\usepackage{amsmath}
\usepackage{amssymb}
\usepackage{amsthm}
\usepackage{array}
\usepackage{setspace}
\usepackage{multicol}
\usepackage[authoryear]{natbib}
\usepackage{epstopdf}
\usepackage{setspace}
\usepackage{fullpage}
\usepackage{bookmark}
\usepackage{subfigure}
\setcounter{MaxMatrixCols}{10}
\renewcommand{\baselinestretch}{1.5}
\setlength{\unitlength}{2em}
\hypersetup{
    colorlinks,%
    citecolor=black,%
    filecolor=black,%
    linkcolor=black,%
    urlcolor=black
}

\begin{document}

\title{Will A Decline in The Corporate Income Tax Rate Create Jobs?\vspace{0.5cm}}

\author{Daphne Chen \\
%EndAName
Florida State University \and Shi Qi\\
%EndAName
Florida State University \vspace{0.5cm} \and Don Schlagenhauf\thanks{%
E-mail: dchen@fsu.edu, sqi@fsu.edu, don.e.schlagenhauf@stls.frb.org. We wish to thank
seminar participants at the Cleveland Fed, the 2012 and 2013 Midwest Macro Meetings, and the 2013 SED Meeting in Seoul for useful suggestions.} \\ Federal Reserve Bank of St. Louis}
\date{\vspace{0.5cm}\today }
\maketitle


\begin{abstract}

We adopt a dynamic stochastic occupational choice model with heterogeneous agents and evaluate the impact of a potential reduction in the corporate income tax on employment. This article finds that a reduction in corporate income tax leads to moderate job creation. In the extreme case, the elimination of the corporate income tax would reduce the non-employed population by 5.4\%. In the model, a reduction in the corporate income tax creates jobs through two channels, one from new entry firms and one from existing firms changing legal organization forms. In particular, the latter accounts for 6/7 of the new jobs created. 

\end{abstract}

\newpage

%\pagenumbering{arabic}

\onehalfspacing

\section{Introduction}

Since its inception in 1909, the appropriateness of the corporate income tax
as a government policy instrument has been questioned. Over the last few years, concerns
with respect to the high corporate income tax rates have focused on the
potential negative effects on employment. Many prominent policy makers and
politicians have suggested that a cut in the corporate income tax rate could
be an engine for job creation. During the 2012 Presidential Election,
Governor Mitt Romney stated that ``if the U.S. Corporate Income tax rate is
reduced ... it makes it easier for small business to keep more of their
capital and hire people. And for me, this is about jobs.'' President Barack
Obama, in responding to this comment, stated ``Governor Romney and I both
agree that our corporate tax rate is too high, so I want to lower it, ...,
taking it down to 25 percent.'' 

In this paper, we ask the question: ``Will a cut in the corporate income tax rate create
jobs?'' The answer to this question may not be as straightforward as suggested by
the politicians. For instance, consider a static representative firm
environment. Allow this firm to have a productive technology that uses
capital and labor as inputs. A corporate income tax is levied against the
total profit of the firm. The firm chooses to maximize profits by equalizing
the marginal returns of both inputs. In the case where both labor and capital expenses are tax deductible, changes in the corporate tax rate does not
distort the relative marginal returns. Hence, this simple
model would predict no change in employment decisions when we adjust the corporate income tax
rate. Furthermore, even if a more sophisticated model predicts an increase in employment, the effect must
be considered quantitatively in relation to others economic issues, such as economic welfare, the wealth distribution, and the long term government budget outlook.

We adopt a dynamic stochastic occupational choice model with heterogeneous agents in a similar fashion to the ``Span of Control'' model presented in \citet{LucasSpan}. Agents are heterogeneous in both productivity and asset holdings in the model economy. They are able to choose between being non-employed, a worker, or firms. There are two important features in our model. The first feature is firm dynamics. Similar to \citet{Hopenhayn1993}, we can accounts for firm entry, exit, growth rate, size distribution as well as job creation, job destruction and job reallocation across firms. A reduction in the corporate income tax increases firm profitability, which can motivate the corporations of new businesses. These newly created firms would in turn generate new jobs in the economy. The jobs generated by new entry firms is potential very important to the long term labor market outlook. \citet{Haltiwanger2013} find that younger firms, if surviving, exhibit substantially higher employment growth as compared to mature firms. In addition, \citet{Haltiwanger2013} document that growing firms contribute the most to job growth in the United States. Therefore, a reduction in the corporate income tax also reduces the likelihood of an existing corporation exiting the market. As a result, the economy would face less job destruction, and enjoy high net job creation.

A second channel is the legal form of organization of firms. In the United States, not all firms are subject to the corporate income tax rate. In fact, only roughly one third of the firms in the U.S. are paying for corporate income tax, but they hire more than half of the workers in the economy. A firm can choose to be either a pass-through business or a C corporation. The pass-through firm can be thought of as an agglomeration of sole proprietorships, partnerships, limited liability firms, S corporations. If a firm files as a pass-through entity, all profits are passed through to owners and are only taxed on individual levels. If a firm files as a C corporation, it is subject to double taxation as the firm profits are subject to the corporate income tax and dividend distributions after paying for corporate income tax are subject to personal income tax. \citet{Goolsbee2004} presents evidence that a lowered corporate income tax reduces the burden of double taxation, and thus can encourage existing pass-through firms to refile as C corporations. 

This disadvantage of a C corporation may be offset by having better access to fund, thus allowing for opportunities of faster growth. This means the unattractive double taxation property of a corporate income tax can be avoided by a new firm filing as a pass-through firm or an existing firm refiling as a pass-through firm. 

These newly formed C corporations, fueled by funds from institutional investors, would expand their operations and hire more employees. Therefore, the margin of switching legal organizations can play a significant role in understanding the employment effect of a corporate income tax rate change.

Both channels of job creation, one from new entry firms and one from existing firms changing legal organization forms, would increase labor demand. The wage would rise in response to the higher labor demand. In addition, the personal income tax would rise in order to compensate for the loss of corporate income tax revenues. The higher wage and the higher personal income tax have opposite effects on the overall economic welfare. Taking into accounts of these counterbalancing forces, this articles finds that a corporate income tax rate of 12\% would optimize economic welfare. Finally, this article takes full advantage of the model's ability of track heterogeneous agent's individual occupation, consumption and saving decisions in a dynamic environment. In doing so, this article finds that 87.2\% of the population would favor lowering the corporate income tax rate to 12\%, while 67.6\% of the population would support the elimination of the corporate income tax.

We calibrate the model to match key statistics such as employment to population ratio, wealth distribution, and firm statistics. Specifically, the model matches the fact that only 24\% of the firms are C corporations who pay corporate income taxes and hire 55\% of the workers. With the calibrated model, we evaluate the impact of a potential reduction in the corporate income tax on employment. This article finds that a reduction in corporate income tax leads to moderate job creation. In the extreme case, the elimination of the corporate income tax would reduce the non-employed population by 5.4\%. In the model, a reduction in the corporate income tax creates jobs through two channels, one from new entrants and one from existing firms changing legal organization form to C corporations. In particular, the latter accounts for 6/7 of the new jobs created.


In addition, changes in the corporate income tax have
implications for the wealth distribution. The corporation income tax rate
can impact agents occupational choices, which in turn affects the asset
accumulation through dynamic saving decisions. As the tax policy changes the
wealth distribution, agents leisure-work decisions can be differentially
impacted. We find that a lowering of the corporate income tax can slightly mitigate the ``Wealth Gap.'' With the elimination of the corporate income tax, the proportion of wealth owned by the top 1\% agents dropped from 17.2\% to 16.4\%. 


A large literature exists on the economic implications of a corporate income
tax. Much of the early research on the corporate income tax focused on
incidence issues as exemplified by \citet{Harberger1962}'s seminal paper as
well as \citet{Ballard1985}, \citet{Feldstein1978}, \citet{Feldstein1980}, %
\citet{Gravelle1989}, and \citet{Shoven1976}. Research by %
\citet{Goolsbee2004} and \citet{Gordon1994} have examined the sensitivity of
the organizational form decision to corporate tax rate changes. Another
issue that has generated research interest is that since dividends are taxed
at both the corporate and individual level, distortions are introduced that
could impact investment efficiency. Some of the research on this issue are %
\citet{Auerbach2002}, \citet{Jensen1986}, \citet{Chetty2005}, and %
\citet{Gourio2010}. Except for employment effects that could emanate from
investment changes, the literature seems to be silent on the employment
impacts from a change in the corporate income tax rate. Our paper presents a
model that not only can be used to address the major research questions
listed above, but also is among the first attempts to answer the question:
``Will a decline in the corporate income tax rate generate jobs?''

[NOTES: need to talk about Ellen's paper, Stony Brooks people's papers, and Goolesbee stuff.]

\section{Model}\label{sec:model}

We consider a dynamic stochastic occupation choice model in a similar
fashion to \citet{LucasSpan} ``Span of Control''. Time is discrete and infinite, indexed by $t=0,1,2,...$. The economy is
consisted with a unit measure of agents, a perfectly competitive financial
intermediary sector, and a government. 

Agents are heterogeneous
in their productive talent and can choose between being non-employed, a
worker, or an entrepreneur. The model allows choices of
business organizational forms. For an entrepreneur, she can decide to run a pass-through
business or a C corporation. Pass-through businesses include sole proprietorships, partnerships, limited liability companies, and S corporations. These pass-through entities are not subject to corporate income taxes and all profits are passed through to the owners. On the hand, a C corporation is subject to corporate income taxes, and is the only organization form that can legally accept funding from institutional investors. Similar to \citet{Hopenhayn1992} and 
\citet{Hopenhayn1993}, the stochastic productivity process generates firm
entry, growth, exit, size distribution as well as job creation, job
destruction and job reallocation across firms.



\subsection{Agents}

Each agent is endowed with one unit of time which can be allocated between
work $n_t$ and leisure $1-n_t$. Agents value consumption $c_t$ and leisure $%
1-n_t$. The per-period utility function $u(c_t, n_t)$ is strictly increasing
and concave in both consumption and leisure. Agents discount the future at
rate $\beta\in[0,1]$.

Agent heterogeneity is reflected through agent states. The agent specific
state are the agent's productivity $z_t$ and her asset level $a_t$. The productivity $z_t$ evolves according to a exogenous
first-order Markov process that is independent across
agents, where $\rho(z'|z)$ is the probability of receiving productivity $z'$ tomorrow given today's productivity is $z$. 
Agents make asset choice decisions $a_{t+1}$ each period. We assume there is a non-borrowing constraint such that $a_{t+1}$ must be non-negative. The assets earn interest at rate $r$.

Agents can make their occupational choices $\chi_t\in\{0,1,2,3\}$. An agent can choose $\chi_t=0$ to be non-employed, $\chi_t=1$ to be an employed worker, $\chi_t=2$ to operate her own pass-through business, or $\chi_t=3$ to manage a C corporation. An agent devotes all of her time to leisure when non-employed $n_t=0$, otherwise she works for a fixed amount of time $\bar{n}$. 

Denote the non-interest earning of an agent with productivity $z_t$, asset level $a_t$ and occupation $\chi_t$ by $E_t(z_t,a_t|\chi_t)$. A non-employed agent receives a lump-sum transfer $b$ from the government, so $E_t(z_t,a_t|\chi_t=0)=b$. An employed worker receives wages rate $w_t$ on her effective labor hours $z_t\bar{n}$, so $E_t(z_t,a_t|\chi_t=1)=w_tz_t\bar{n}$. If an agent decides to run a business, either a pass-through business or a C corporation, the firm's produces output according to $F(z_t,k_t,l_t)$. This production function is increasing in productivity $z_t$, capital $k_t$ and labor $l_t$. The input prices are $r_t$ for capital and $w_t$ for labor. An operating business incurs a fixed cost $c_f$ each period regardless of the business scale. A firm can avoid paying the fixed cost by shutting down operation in a period.

If an agent chooses to operate a pass-through business or $\chi_t=2$, her
firm is subject to a collateral constraint $k_t\leq a_t$. In other words,
she cannot borrow more capital stock than her current asset holding. An
agent operating a pass-through business receives the maximized profit from
her firm $\pi(z_t,a_t)$ constrained by her own asset savings $a_t$, where 
\begin{eqnarray*}
\pi_t(z_t,a_t)&=&\max_{k_t,l_t}F(z_t,k_t,l_t)-(r_t+\delta)k_t-w_tl_t-c_f\\
\text{subject to } && 0<k_t\leq a_t
\end{eqnarray*}
Let the optimal labor choice be $l_t(z_t,a_t)$, and the optimal capital choice
be $k_t(z_t,a_t)$. So $E_t(z_t,a_t|\chi_t=1)=\max\{0,\pi_t(z_t,a_t)\}$

An agent can also choose to manage a C corporation or $\chi_t=3$. A
C corporation is funded by financial intermediaries. We assume financial
intermediaries have ``deep pockets'' and C corporations are therefore not
subject to any collateral constraints. A C corporation generates profit $\pi^*(z_t)$, where 
\begin{equation*}
\pi_t^*(z_t)=\max_{k_t,l_t}F(z_t,k_t,l_t)-(r_t+\delta)k_t-w_tl_t-c_f
\end{equation*}
Let the optimal labor choice be $l_t^*(z_t)$, and the optimal capital choice
be $k_t^*(z_t)$. A C corporation's profit $\pi_t^*(z_t)$ is subject to corporate income tax rate $\tau^c$.

Because the agent's productivity or managerial talent determines the profit
of a corporation, she can negotiate a compensation package with the
corporation based on her productivity level $z_t$. We assume she takes a $%
\phi_t(z_t)$ share of the firm's after-tax profit as her compensation. She
therefore receives a compensation payment each period $E_t(z_t,a_t|\chi_t=3)=\max\{0,\phi_t(z_t)(1-%
\tau^c)\pi_t^*(z_t)\}$.

Because of the collateral constraint $k_t\leq a_t$ faced by pass-through
firms, it is trivial to show that $\pi_t^*(z_t)\geq \pi_t(z_t,a_t)$ for a given $%
z_t$ and $a_t$. The collateral constraint is important for understanding the
key trade-off between choosing a pass-through firm structure versus a
corporate firm structure. A binding collateral constraint, together with a
non-borrowing constraint, may limit an individual agent's ability to finance
the operation of her firm. This fact gives incentive for agents to seek
outside funding by inviting corporate investment from financial
intermediaries, even when C corporations are faced with the downside of
double taxation.

\subsection{Financial Intermediary}
There exists a representative competitive financial intermediary. It can only invest in C corporations. It incurs a transaction cost $c_e$ to invest in a C corporation. Provided that the financial intermediary has full information of firm's productivity $z_t$, it will supply capital $k^*_t(z_t)$ to the production in exchange for shareholdings of the firm. The after-tax corporate profits are split such that the agent receives $\phi_t(z_t)$ shares of the after-tax profits and the financial intermediary receives the remaining $1-\phi_t(z_t)$ shares. 

\subsection{Government}
The government collects two types of taxes: personal income tax and corporate income tax. The tax revenues are used to fund the lump-sum transfer $b$ to every non-employed agent and an exogenous government spending $G$. The government runs a balanced budget.

On the individual level, all agents' non-interest earnings $E_t(z_t,a_t|\chi_t)$ and
their interest earnings $r_ta_t$ are subject to personal income tax $\tau^p$. On the corporate level, all C corporations pay the entry transaction cost $c_e$, which is tax deductible. So the net corporate profits $\pi^*_t(z_t)-c_e$ is subject to the corporate income tax $\tau^c$.


\subsection{Timing of the Events}
The timing of the events within a period is given as follows:

\begin{enumerate}
\item An agent enters a period with productivity $z_t$ and asset level $a_t$;
\item The agent makes an occupational decisions $\chi_t$ to be non-employed, a worker, a pass-through business entrepreneur, or a C corporation entrepreneur;
\item Production occurs. All agents receive their respective earnings; 
\item The government taxes personal and corporate income to fund transfer and government spending;
\item Consumption and saving decisions takes place; 
\item Agents draw new productivity shocks, and the period ends. 
\end{enumerate}
\section{Equilibrium}

We define the general equilibrium for the dynamic stochastic occupational choice model described in Section \ref{sec:model}. Because we focus on a stationary equilibrium, all time subscripts $t$ are suppressed.

\subsection{Agent's Problem}

Let $V(z,a)$ be the value functions for an agent with productivity $z$ and asset level $a$. Let $W^\chi(z,a)$ be the value functions given an occupational choice $\chi$. 

\subsubsection{Non-employed Agents}
A non-employed agent receives a lump-sum transfer $b$ from the government. The government transfer $b$ and interest income $ra$ are subject to personal income tax at rate $\tau^p$. She maximizes her lifetime utility by making asset choice decisions $a'$ subject to the budget constraint, consumption non-negativity, and non-borrowing constraint. Her value function is given by
\begin{eqnarray}
W^0(z,a)&&=\max_{ c,a^{\prime} }\ \  u(c,1)+\beta E_{z^{\prime}|z}
V(z^{\prime},a^{\prime})  \label{eqn:W0}\\
\text{subject to}&&\nonumber\\
&& c = (1-\tau^p)(b+ra)+a-a^{\prime} \nonumber\\
&& c \geq 0;\quad a^{\prime}\geq 0\nonumber
\end{eqnarray}

\subsubsection{Employed Workers}
An employed worker receives a wage $w$ on her effective labor $z\bar{n}$. Her earnings $wz\bar{n}$ and interest income $ra$ are subject to personal income tax at rate $\tau^p$. She receives utility from consumption $c$ but dis-utility from work $\bar{n}$. Her value function is given by
\begin{eqnarray}
W^1(z,a)&&=\max_{ c,a^{\prime} } \ \ u(c,1-\bar{n})+\beta
E_{z^{\prime}|z} V(z^{\prime},a^{\prime})  \label{eqn:W1}\\
\text{subject to}&&\nonumber\\
&& c = (1-\tau^p)(w z \bar{n}+ra)+a-a^{\prime}\nonumber \\
&& c \geq 0;\quad a^{\prime}\geq 0\nonumber
\end{eqnarray}

\subsubsection{Pass-through Business Entrepreneurs}
An entrepreneur under the pass-through organizational form must self finance the firm operation and can receive profits $\pi(z,a)$ given his productivity $z$ and asset $a$. The profit is not subject to corporate income tax. Her business earnings and interest income $ra$ are subject to personal income tax at rate $\tau^p$.
\begin{eqnarray}
W^2(z,a)&&=\max_{ c,a^{\prime} } \ \ u(c,1-\bar{n})+\beta
E_{z^{\prime}|z} V(z^{\prime},a^{\prime})   \label{eqn:W2}\\
\text{subject to}&&\nonumber\\
&& c = (1-\tau^p)(\max\{0,\pi(z,a)\}+ra)+a-a^{\prime}\nonumber \\
&& c \geq 0;\quad a^{\prime}\geq 0\nonumber
\end{eqnarray}

\subsubsection{C Corporation Entrepreneurs}
An entrepreneur runs a C corporation with institutional investors. The C corporation faces no financial constraints, but some part of the profits has to be shared with the investors. The profits are subject to double taxation on both corporate and individual levels. The entrepreneur's corporate earnings and interest income $ra$ are subject to personal income tax at rate $\tau^p$.
\begin{eqnarray}
W^3(z,a)&&=\max_{ c,a^{\prime} } \ \ u(c,1-\bar{n})+\beta
E_{z^{\prime}|z} V(z^{\prime},a^{\prime})  \label{eqn:W3}\\
\text{subject to}&&\nonumber\\
&& c = (1-\tau^p)(\max\{0,\phi(z)(1-\tau^c)\pi^*(z)\}+ra)+a-a^{\prime} \nonumber\\
&& c \geq 0;\quad a^{\prime}\geq 0\nonumber
\end{eqnarray}

In the beginning of the period, an agent learns her productivity $z$ and asset position $a$, and she maximizes her lifetime utility by making the static occupational choice $\chi$. The value function is given by
\begin{equation}
V(z,a)=\max\{ W^0(z,a), W^1(z,a), W^2(z,a), W^3(z,a)\}\label{eqn:VW}
\end{equation}
The solution to this agent's problem gives the optimal occupational choice decisions $\chi(z,a)$, consumption choice $c(z,a)$, asset choice decisions $a'(z,a)$, and the labor supply decision $n(z,a)$. 

\subsection{Financial Intermediary}
The financial intermediary is subject to participation
constraint, where it would only choose to finance a C corporation if 
\begin{eqnarray*}
(1-\phi_t(z_t))(1-\tau^c)\pi^*_t(z_t)&\geq& (1-\tau^c) c_e\\
(1-\phi_t(z_t))\pi^*_t(z_t)&\geq& c_e
\end{eqnarray*}
where $c_e$ is the fixed transaction cost. Because $c_e$ determines whether financial intermediary would fund a C corporation, and $c_e$ is tax deductible, the financial intermediary investment decision does not depend the size of the corporate income tax.

In addition, because the financial intermediary is competitive, the profit to invest a C corporation with talent $z$ is non-positive such that
\begin{equation*}
(1-\phi_t(z_t))\pi^*_t(z_t)\leq c_e.
\end{equation*}
Together with the participation constraint, a financial intermediary must make zero
profit in equilibrium. Therefore,
\begin{equation}
(1-\phi_t(z_t))\pi_t^*(z_t) = c_e. \label{eqn:phi}
\end{equation}
Because the transaction cost $c_e$ is fixed, when a firm is more productive, fewer shares are needed to make the financial intermediary willing to invest. This means the fraction of shares the firm can keep to itself $\phi_t(z_t)$ is increasing in productivity $z$. Furthermore, if the productivity is extremely low, the financial intermediary may not be able to cover the transaction cost even when offered all the shares. In another word, the participation constraint is violated. We can define a cutoff productivity $\underline{z}>0$ such that 
\begin{equation*}
\pi_t^*(\underline{z}) = c_e.
\end{equation*}
The financial intermediary will refuse to invest at all for any firm that has productivity lower than this cutoff level, so $\phi(z)=0$, $\forall z\leq\underline{z}$.


\subsection{Distributions}

Let $\mu(z,a)$ denote the invariant cross-sectional distribution measures of
agents with productivity $z$ and asset $a$. The evolution of the distribution depends on the endogenous asset choice $a^{\prime}(z,a)$ and the exogenous Markov process of the productivity $z$. For any set of future asset levels $\mathcal{A}$, and any future productivity $z'$, the following is satisfied.
\begin{eqnarray*}
&&\mu(z^{\prime},\mathcal{A})= \int_{z,a}
1_{\{a'(z,a)\in \mathcal{A}\}} \rho(z^{\prime}|z)  \mu(dz,da)
\end{eqnarray*}

\subsection{Government's Budget}

The government collects tax revenues from agents and C corporations to finance the lump-sum transfers to non-employed agents and a government spending $G$. 
We denote the individual incomes subject to personal income tax at rate $\tau^p$ given productivity $z$ and wealth $a$ as $y(z,a)$ such that
\begin{equation}
y(z,a)=E(z,a|\chi(z,a))
\end{equation}
The total government revenue collected
from individual agents is 
\begin{equation*}
R^p=\tau^p \int_{z,a} y(z,a)\mu(dz,da)
\end{equation*}
The total government revenue collected from C corporations is 
\begin{equation*}
R^c=\tau^c\int_{z,a} 1_{\{\chi(z,a)=3\}}(\pi^*(z)-c_e) \mu(dz, da)
\end{equation*}
The total non-employment transfer is 
\begin{equation*}
B=\int_{z,a} 1_{\{\chi(z,a)=0\}} b\mu(dz, da)
\end{equation*}
Because the government runs a balanced budget, the government expenditure has to be equal to the government tax revenues. 
\begin{equation}
G+B=R^p +R^c \label{eqn:govbudget}
\end{equation}

\subsection{Labor Market Clearing Condition}
Equilibrium wage clears the labor market.\footnote{The model has a fixed interest rate $r$. We assume that financial intermediaries have access to foreign capital market. Therefore, in the domestic capital market, $K^D\geq K^S$.} The effective labor supply from an employed worker is his productivity $z$ times the hours worked $\bar{n}$. We aggregate over all employed workers to obtain the total labor supply 
\begin{equation*}
L^s=\int_{z,a} 1_{\{\chi(z,a)=1\}} z\bar{n}\mu(dz,da).
\end{equation*}
When an entrepreneur runs a pass-through business, she demands labor $l(z,a)$ according to her productivity $z$ and asset position $z$. When an entrepreneur runs a C corporation, the labor demand $l^*(z)$ depends only on productivity. Aggregating labor demand across entrepreneurs over different organizational forms, the total labor demand is 
\begin{equation*}
L^d=\int_{z,a} \bigg[ 1_{\{\chi(z,a)=2\}} l(z,a)
+1_{\{\chi(z,a)=3\}} l^*(z)\bigg] \mu(dz,da)
\end{equation*}
The excess labor market supply is defined as 
\begin{equation}
\Delta_L=L^s-L^d  \label{eqn:deltaL}.
\end{equation}
In equilibrium, labor market clears with zero excess supply. 


\subsection{Equilibrium Definition}

A steady-state equilibrium consists of a set of agents' decision rules $\chi^*(z,a)$, $c^*(z,a)$, $a'^*(z,a)$, $n^*(z,a)$, a profit sharing rule $\phi^*(z)$, a wage rate $w^*$, a corporate income tax rate $\tau^c$, a personal income tax rate $\tau^p$, and a distribution $\mu^*(z,a)$ such that given the exogenous government spending $G$, non-employment transfer $b$, and risk-free rate return $r$
\begin{enumerate}
\item The decision rules $\chi^*(z,a)$, $c^*(z,a)$, $a'^*(z,a)$, and $n^*(z,a)$ solve the agent's optimization problem as in \eqref{eqn:W0}, \eqref{eqn:W1}, \eqref{eqn:W2}, \eqref{eqn:W3}, and \eqref{eqn:VW}; 
\item the profit sharing rule $\phi^*(z)$ satisfies the zero profit condition for the financial intermediary as in \eqref{eqn:phi};
\item labor market clears, $\Delta_L=0$; %, $\Delta_K=0$, and $\Delta_Y=0$;
\item the government runs a balanced budget as in \eqref{eqn:govbudget};
\item the distribution $\mu^*(z,a)$ reproduces itself.
\end{enumerate}

\section{Calibration}

A model period is one year. Because the corporate and personal income taxes are assumed to be flat rate,
we map them with the average effective tax rate from data. For corporate
income tax, according to U.S. Bureau of Economic Analysis (BEA), the average
corporate income tax revenues, which include taxes on corporate income paid
to federal, state, and local government, as percentage of corporate income
from 2001 to 2008 is 25.7 percent. The personal income tax is set to be 20 percent. 

The discount rate $\beta$ is set to be 0.96. The utility function is assume to be separable
in consumption and leisure and takes the following form: 
\begin{equation}
u(c,l)=\frac{c^{1-\alpha_c}}{1-\alpha_c}+\psi\frac{l^{1+\alpha_n}}{1+\alpha_n}.
\end{equation}
The labor supply decisions are discrete. Agents either stay unemployed or
work full-time as a worker or an entrepreneur. The full-time work amount $%
\bar{n}$ is set to be 0.45, corresponding to 45 hours a week. The risk free
interest rate is set to be 0.01 consistent with the average of the long-term
U.S. inflation indexed government securities. See Table \ref{tab_para1} for the set of parameters that we
calibrate independently from data.

\begin{table}[!ht]
\caption{\sc Parameters Calibrated Independently}
\label{tab_para1}
\begin{center}
\begin{tabular}{lcc}
\hline\hline
Description &  Parameter &      Value \\
\hline
Corporate Income Tax Rate &   $\tau^c$ &       0.26 \\

Personal Income Tax Rate &   $\tau^p$ &       0.20 \\

Discount Rate &    $\beta$ &       0.96 \\

Risk-free Interest Rate &        $r$ &       0.01 \\

Capital Depreciation Rate &   $\delta$ &       0.06 \\

Full-time Work Hours &  $\bar{n}$ &       0.45 \\
\hline\hline
\end{tabular}  
\end{center}
\end{table}

For all other parameters, we calibrate their values through a moment
matching exercise, summarized in Table \ref{tab_para2}. The logarithm of the productivity $z$ follows an AR(1) process with
autocorrelation $\rho_z$ and standard deviation $\sigma_z$, $%
\log(z^{\prime})=\rho_z\log(z)+\varepsilon$ where $\varepsilon\sim
N(0,\sigma_\varepsilon)$. The autocorrelation and standard deviation of the log productivity are calibrated to be 0.879 and 0.198. As for the preferences, the constant relative risk aversion coefficient for consumption is 3.251. The constant and power parameters for the leisure are 0.171 and 0.142 respectively. 

The production function is assumed to take the following form, $F(z,k,n)=zk^\gamma n^\theta$. The capital and labor share in the production function are 0.223 and 0.485
respectively. The capital depreciation rate $\delta$ is set to be 10 percent from
standard estimation. The fixed cost for operating a business $c_f$ is 1.698. The entry
cost paid to the financial intermediary for fund access $c_e$ is 4.858. Finally, the
lump-sum transfer $b$ received by non-employed workers is 0.248.

\begin{table}[!ht]
\caption{\sc Parameters Calibrated Jointly in Equilibrium}
\label{tab_para2}
\begin{center}
\begin{tabular}{lcc}
\hline\hline
Description &  Parameter &      Value \\
\hline
Persistence for Productivity &   $\rho_z$ &      0.879 \\

Standard Deviation for Productivity & $\sigma_z$ &      0.198 \\

Constant Leisure Parameter  &   $\psi$ &      0.171 \\

Power Leisure Parameter  & $\alpha_n$ &      0.142 \\

CRRA Coefficient on Consumption & $\alpha_c$ &      3.251 \\

Capital Share in Production Function&   $\gamma$ &      0.223 \\

Labor Share in Production Function&   $\theta$ &      0.485 \\

Firm Fixed Cost &      $c_f$ &      1.698 \\

Entry Cost for Access to Fund &      $c_e$ &      4.858 \\

Non-employment Lump-sum Transfer &        $b$ &      0.248 \\
\hline\hline
\end{tabular}  
\end{center}
\end{table}

These parameters are estimated jointly such that we match data and
model moments. First of all, we match a set of key employment statistics. According to Bureau of Labor Statistics (BLS), the fraction of civilian employed between 25 and 64 was around 75 percent. Therefore, we targeted the non-employed fraction in the population to be 25 percent. From U.S. Census Bureau 2007, we have the number of firms and the numbers of workers hired by each legal form of organizations. This provides key statistics regarding the employment distribution. The fraction of C corporations is 23.9 percent, and the employment share by C corporations is 54.63 percent. The jobs created from entry firms account for 36.2 percent of total new jobs. 

According to Budget of the United States Government, corporate income tax accounts for 9 percent of the federal tax revenues. The labor share in output is targeted at 0.6. In order to have a sensible labor response to the change in tax policy, we target the macro labor supply elasticity to be 3 in line with the standard estimation between 2 and 5. Finally, we match the key wealth statistics. The wealth gini index is 0.82 according to 2007 Survey of Consumer Finance (SCF). We also include the percentage wealth held by top households. See Table \ref{tab_moment} for the set of moments. 

\begin{table}[!ht]
\caption{\sc Data and Model Moments}
\label{tab_moment}
\begin{center}
\begin{tabular}{lcc}
\hline\hline
Statistics &       Data &      Model \\
\hline
Non-employment Fraction in Population &      0.250 &      0.239 \\

Fraction of C Corporations &      0.239 &      0.238 \\

Employment Fraction by C Corporations &      0.546 &      0.564 \\

Fraction of Jobs Created by Entry Firm &      0.362 &      0.271 \\

Corporate Income Tax to Total Tax Revenues Ratio &      0.090 &      0.122 \\

Labor Share of Income &      0.666 &      0.637 \\

Labor Supply Elasticity &      3.000 &      2.976 \\

Wealth Gini Index &      0.820 &      0.803 \\

Percentage of Wealth in Top $60\%$ &      0.990 &      0.988 \\

Percentage of Wealth in Top $40\%$ &      0.950 &      0.954 \\

Percentage of Wealth in Top $20\%$  &      0.830 &       0.860 \\

Percentage of Wealth in Top $10\%$ &      0.710 &      0.695 \\

Percentage of Wealth in Top $1\%$ &      0.340 &      0.172 \\

\hline\hline
\end{tabular}  
\end{center}
\end{table}

\section{Results}

\subsection{Benchmark}

Since the labor choice is discrete, agents receive the same disutility once they work, and the decision of whether to be a worker or an entrepreneur only depends on the current period non-interest income graphed in Figure \ref{fig:y}. If they become a worker, they receive earnings $wz\bar{n}$ from work shown in solid line, which is linear in productivity $z$. If they run a pass-through business, they produce subject to collateral constraint and receive profits $\pi(z,a)$ shown in dash line. Finally, if they run a C corporation, they receive dividends according to their shares $\phi(z)d(z)$ shown in dotted line.
\begin{figure}[!ht]
\centering
\caption{\sc Per-Period Non-Interest Income for Employed Agents}
\label{fig:y}\includegraphics[scale=0.6]{payoff_func.eps}
\end{figure}
Agents with lower productivity are more likely to be a worker because they can receive the market wage. When their productivity gets higher, they start a business. Agents with relatively low productivity are less constrained due to low demand for capital, and they will choose to have a pass-through business to avoid double taxation. Agents with very high productivity would demand much more capital and are willing to pay corporate income tax in order to have better access to fund. Finally, agents need to evaluate whether or not they would like to stay home. For agents with very low productivity, they can receive non-employment lump-sum transfer $b$ that is independent of their ability. Because agents also enjoy leisure when they choose not to work, for those low ability ones, their incentives to stay home are higher when they are wealthier. The equilibrium occupation choice decisions are graphed in Figure \ref%
{fig:oc}.

\begin{figure}[!ht]
\centering
\caption{\sc Occupation Choices in Benchmark Model}
\label{fig:oc}\includegraphics[scale=0.6]{ochoice.eps}
\end{figure}


\subsection{Policy Experiment}

To answer the question whether there will be more jobs when we lower
corporate income tax, we take the benchmark parameters calibrated to the U.S. data but reduce the corporate income tax rate and see how the equilibrium changes. 

From Figure \ref{fig:ochoice_exp}, we can see when we lower the corporate income tax, some pass-through entity will now switch to be C corporations because the tax burden is lower, while some pass-through entity will switch to be workers because the wages are higher due to higher labor demand. More non-employed workers will turn to be workers also because of higher wages. Notice that even when we eliminate the corporate income tax, there are still agents who choose to be pass-through entity instead of C corporations, because if their productivity is not high enough, they need to give a larger percentage of shares to financial intermediary for fund access. 

\begin{figure}[!ht]
\centering
\caption{\sc Occupation Choices in Policy Experiment}
\includegraphics[scale=0.6]{ochoice_exp.eps}\label{fig:ochoice_exp}
\end{figure}

Since C corporations are subject to lower tax rate, as evident from Figure \ref{fig:firmstat}, the fraction of C corporations among all firms is increasing when we lower corporate tax rate. Notice that because we are interested in the reduction in corporate income tax rate, the horizontal axis is decreasing in the rate. Since we hare more C corporations in the economy, and C corporations demand more labor conditional on the same productivity because they have unconstrained capital, the percentage of workers hired by C corporations increases when corporate income tax rate lowers. 

\begin{figure}[!ht]
\centering
\caption{\sc Firm Statistics in Policy Experiment}\label{fig:firmstat}
\subfiguretopcaptrue
\subfigure[\sc C Corporation Fraction]{\includegraphics[scale=0.4]{cfrac.eps}\label{fig:cfrac}}
\quad
\subfigure[\sc Employment Fraction by C Corp.]{\includegraphics[scale=0.4]{cempfrac.eps}\label{fig:cempfrac}}
\end{figure}

In the top panel of Figure \ref{fig:exp}, we graph the equilibrium output and wage when we lower the corporate income tax rate. When corporate income tax rate goes down, total output goes up since it encourages households to be entrepreneurs. Labor demand for production increases, so does the equilibrium wages. Since the equilibrium wages are higher, the wealth gini decreases, because as we can see from Figure \ref{fig:assetdistchg} where there are households moving from the lower end of the wealth distribution toward the right, which makes the inequality lower. Finally, the personal income tax $\tau^p$ is increasing when we lower the corporate income tax rate in order to balance government's budget. 

\begin{figure}[!ht]
	\centering
	\caption{\sc Tax $\tau^p$, Wealth Gini, Output and Wage in Policy Experiment}\label{fig:exp}
	\subfiguretopcaptrue
	\subfigure[\sc Output $Y$]{\includegraphics[scale=0.4]{output.eps}\label{fig:output}}
	\quad
	\subfigure[\sc Wage $w$]{\includegraphics[scale=0.4]{wage.eps}\label{fig:wage}}
	\subfiguretopcaptrue
	\subfigure[\sc Wealth Gini]{\includegraphics[scale=0.4]{gini.eps}\label{fig:gini}}
	\quad
	\subfigure[\sc Personal Income Tax $\tau^p$]{\includegraphics[scale=0.4]{ptax.eps}\label{fig:taup}}
\end{figure}


\begin{figure}[!ht]
     \centering
     \caption{\sc Asset Distribution Changes in Policy Experiment}\label{fig:assetdistchg}
\subfiguretopcaptrue
\subfigure[\sc Benchmark Asset Distribution]{\includegraphics[scale=0.4]{assetdist.eps}}\\
\subfiguretopcaptrue
\subfigure[\sc Asset Dist. Change ($\tau^c=12\%$)]{\includegraphics[scale=0.4]{assetdistchg12.eps}\label{fig:assetdistchg12}}\\
\subfiguretopcaptrue
\subfigure[\sc Asset Dist. Change ($\tau^c=0\%$)]{\includegraphics[scale=0.4]{assetdistchg0.eps}\label{fig:assetdistchg0}}
\end{figure}

Figure \ref{fig:nonemprate} graphs the fraction of non-employed households in the economy when we lower corporate income tax rate. When there is a decline in corporate income tax rate, we can see fewer households are non-employed. This suggests that we can generate jobs by cutting corporate income tax. When we completely eliminate corporate income tax, the non-employed fraction can be reduced by 5.4\%. 

\begin{figure}[!ht]
\centering
\caption{\sc Non-employment Rate in Policy Experiment }
\includegraphics[scale=0.6]{nonemprate.eps}\label{fig:nonemprate}
\end{figure}

\subsection{The Importance of LFO Selection}

Figure \ref{fig:nonemprate_nos} graphs the relative change in non-employment rates when we lower the corporate income tax. The black solid line represents our benchmark results in an economy where firms can decide whether to operate a C corporation or a pass-through business. As we can see from the graph, in the extreme case when there is no tax liability on the corporate levels, the non-employment rate can drop by more than 5 percent. 

In order to see the importance of the corporate type selection, suppose we take a model where entrepreneurs can only run C corporations and must pay the corporate income tax. If we calibrate this model to our economy treating every firm as C corporation, then when we lower the corporate income tax rate, the relative change in the non-employment rates is shown as the red dash line. The effect of changes in corporate income tax on non-employment rates is minimal in an economy without distinguishing the difference between pass-through entities and C corporations. The reason is because pass-through business owners do not pay corporate income taxes but due to their restricted access to fund, conditional on the same productivity, they are more likely to demand less workers. When we lower the corporate income tax, in an economy without firm organizational form choices, we miss the margin where firms would hire more when they switch from pass-through entities to C corporations.


Since we no longer have the option to run pass-through businesses which tend to be smaller and higher fewer workers, all firms must operate as C corporations and that implies the labor demand would be higher given the same productivity. When we re-calibrate to match the key statistics such as the non-employment rate, on the worker's side, they value leisure more in order to keep the labor supply down. To be more specific, the constant parameter on leisure increases from 0.17 to 0.25. On the firm side, in order to clear the market and keep the labor demand down, the costs of operating a firm are calibrated to be more expensive in order to deter firm creation and then keep employment low. In particular, the fixed cost goes up from 1.69 to 4.58 and the entry cost for access to fund increases from 4.86 to 6.47.

To better understand the differences in how equilibrium employment responds to the changes in corporate income tax rates in the two economies, we consider two things: one is the structural change about whether firms have corporate type choices, and the other is the parameter change since both economies are calibrated independently to match U.S. statistics. To decompose the importance of each change, we consider an economy where all firms must run as C corporations using the benchmark parameters without recalibration, and the results are shown as the blue dotted line. As evident from the graph, the changes in non-employment rates are still underestimated compared with our benchmark results, although the effects are larger than the re-calibrated economy. Therefore, the difference between the benchmark results and this one is due to structural change, while the difference between this one and the results from re-calibrated economy with only C corporation is due to parameter change.

\begin{figure}[!ht]
\centering
\caption{\sc Non-employment Rate in Policy Experiment Comparisons }
\includegraphics[scale=0.6]{nonemprate_nos.eps}\label{fig:nonemprate_nos}
\end{figure}

\subsection{Welfare Analysis}

Lowering the corporate income tax has an ambiguous effect on welfare. On the one hand, C corporations are no longer subject to corporate income tax and can keep more of their profits. On the other hand, the reduction of tax revenue from corporations has to be compensated by the increase in personal income tax. The increase in personal income tax affects every agent in the economy. We quantify the effect by calculating consumption equivalent welfare.

We start from the cross-sectional distribution in our benchmark economy, and
we ask each agent what is the percentage of consumption they are willing to
give up in all contingencies in all future periods in order to live in the
economy after tax policy change. Let $V(z,a;\tau^c)$ be the lifetime utility
for an agent in state $(z,a)$ in an economy with corporate income tax rate $\tau^c$. The consumption equivalent welfare $\eta(z,a;\tau^c)$ is given by 
\begin{equation}
\eta(z,a;\tau^c)=\left(\frac{V(z,a;\tau^c)}{V(z,a;\tau^c_{bench})}\right)^{\frac{1}{1-\alpha_c}}-1
\end{equation}
If the consumption equivalent welfare $\eta(z,a;\tau^c)$ is positive, it means the
agent is better off in the counterfactual economy with corporate income tax rate $\tau^c$. If it is negative, it means the agent has welfare loss after the policy change. 

Figure \ref{fig:welfare} graphs the average consumption equivalents welfare for different proposed corporate income tax rates. When we start decreasing the corporate income tax, agents benefit from having higher wages and corporations benefit from having higher after-tax profits, and benefits outweigh the cost of higher personal income tax. However, if we continue to decrease the corporate income tax rate, although we remain to have welfare gains from the tax policy change, the gains turn smaller, that is when the cost of having everyone to pay higher personal income tax outweighs the benefits of higher wages and higher profits. It is clear from the graph that the welfare gains have a reverse U shape in corporate income tax rate, and it reaches the maximum at the rate of 12\%. 

Table \ref{tab_CE} shows the consumption equivalent welfare by each
occupation when we consider two alternative tax scenarios. In the first case, we consider the corporate income tax rate that maximizes average consumption equivalents welfare. In the second case, we consider the elimination of corporate income tax.
 
C corporations contribute the most welfare gains in both cases because they enjoy lower corporate tax liability. Workers are also benefited by the policy due to higher wages. On the other hand, to compensate for the loss in corporate tax revenue, the personal income tax has to be increased to balance the government budget. The pass-through businesses are not benefited by the lower corporate
tax because they are not subject to double taxation, but now they have to
pay higher personal income tax and pay higher wages to their workers, so
they are worse off. Non-employed workers have higher incentives to switch to workers due to higher wages but also suffer from higher personal income tax. So the two tax policies predict different welfare results for non-employed workers. When the corporate income tax rate is 12\%, non-employed workers are slightly better off because they enjoy higher wages when they switch occupations but don't suffer too much from higher personal income tax since corporations still share some tax liabilities in the economy. However, when we completely eliminate the corporate income tax, non-employed workers become worse off because of the burdens from personal income tax liability. 

The overall welfare can be calculated by integrating over individual welfare 
$\eta(z,a;\tau^c)$ weighted by the distribution measure $\mu(z,a;\tau^c_{bench})$ from the benchmark economy. The average welfare gain is 0.23\% when corporate income tax rate is 12\% and 0.08\% when we eliminate the corporate income tax. Although the average welfare numbers are small, a large majority of the economy is in favor of the policy change. Specifically, 87.27 percent of all households would prefer the optimal corporate income tax rate at 12\% and 67.61 percent would prefer eliminating corporate income tax from the economy.  

\begin{figure}[!ht]
\centering
\caption{\sc CE Welfare in Policy Experiment }
\includegraphics[scale=0.6]{welfare.eps}\label{fig:welfare}
\end{figure}

\begin{table}[!ht]
\caption{\sc Welfare by Occupation}
\label{tab_CE}
\begin{center}
\begin{tabular}{lr@{.}lr@{.}lr@{.}lr@{.}lr@{.}l}
\hline\hline
\multicolumn{ 1}{c}{Occupation} & \multicolumn{ 2}{c}{Non-} & \multicolumn{ 2}{c}{Worker} & \multicolumn{ 2}{c}{Pass-} & \multicolumn{ 2}{c}{C Corp} & \multicolumn{ 2}{c}{Overall} \\

\multicolumn{ 1}{c}{} & \multicolumn{ 2}{c}{employed} &   \multicolumn{ 2}{c}{} & \multicolumn{ 2}{c}{through} &   \multicolumn{ 2}{c}{} &   \multicolumn{ 2}{c}{} \\
\hline
Proportion of Agents &          0 &       2386 &          0 &       7304 &          0 &       0236 &          0 &       0074 &          1 &        000 \\
\hline
                                                                            \multicolumn{ 11}{l}{\bf $\tau^c=12\%$} \\

Average \% CE Welfare Gain &          0 &         03 &          0 &         29 &         -0 &         34 &          2 &         12 &          0 &         23 \\

\% of Agents in Favor of Policy Change &         66 &         23 &         96 &         45 &         12 &         21 &        100 &         00 &         87 &         27 \\
\hline
                                                                                    \multicolumn{ 11}{l}{\bf $\tau^c=0\%$} \\

Average \% CE Welfare Gain &         -0 &         26 &          0 &         19 &         -0 &         74 &          3 &         54 &          0 &         08 \\

\% of Agents in Favor of Policy Change &          6 &         85 &         89 &         05 &          8 &         46 &        100 &         00 &         67 &         61 \\
\hline\hline
\end{tabular}       
\end{center}
\end{table}

\subsection{The Role of Government Budget Balancing}
In our benchmark economy, we assume that the government budget is balanced. Therefore, if the total tax revenue is reduced once we lower the corporate income tax rate, that means the personal income tax rate has to be increased in order to cover the exogenous government spending and non-employment lump-sum transfer. In this section, we relax the assumption that the exogenous government spending has to be met and investigate how that would affect our results and welfare analysis. To be specific, when we run the policy experiment, we hold the personal income tax rate the same, if the tax revenues can not cover the exogenous government spending after the non-employment transfers are paid, then the government would cut the spending in order to balance the budget.

Figure \ref{fig:noptax} graphs the experiment results holding the personal income tax rate same as the benchmark economy. As we lower the corporate income tax rate, the government tax revenues decrease if we are not allowed to increase the taxes on individual levels. That means the government spending has to be cut back in order to balance the budget. In Figure \ref{fig:gbar_noptax}, it is clear that government spending has to decrease as we reduce the corporate income tax rate. However, since government spending does not enter household's utility in any way, it is assumed to be thrown into the ocean, we no longer have the welfare going in ambiguous direction when we lower the corporate income tax, the welfare is unambiguously monotonically increasing when we lower corporate income tax rate without increasing the personal income tax. 

In the bottom panel of the same graph, when we lower the corporate income tax rate, the equilibrium wages increase in our benchmark economy because there is more labor demand. When we hold personal income tax rate fixed, the wages still increase when we lower corporate income tax rate but the response is slightly higher than the benchmark, which results from an increase in leisure demand when households pay less personal income taxes and have higher wealth. The labor supply is lower, hence the equilibrium wages to clear the market has to respond little bit more than the original policy experiments.

Finally, Figure \ref{fig:noemprate_noptax} compares the key results for non-employment rate changes with and without holding personal income tax rate fixed. We can see this assumption does not affect our results much, so our benchmark analysis is robust regardless of how the government budget constraint is balanced, although when holding the personal income tax rate, it initially estimates a lower response but later a higher response than our original policy experiments. That illustrates a tradeoff between lower labor supply from household enjoying leisure and higher labor demand from more incentives for firms to run as C corporations. 

\begin{figure}[!ht]
\centering
\caption{\sc Policy Experiment if Fixing $\tau^p$}\label{fig:noptax}
\subfiguretopcaptrue
\subfigure[\sc Government Spending]{\includegraphics[scale=0.4]{gbar_noptax.eps}\label{fig:gbar_noptax}}
\quad
\subfigure[\sc CE Welfare]{\includegraphics[scale=0.4]{welfare_noptax.eps}\label{fig:welfare_noptax}}
\subfiguretopcaptrue
\subfigure[\sc Wage]{\includegraphics[scale=0.4]{wage_noptax.eps}\label{fig:wage_noptax}}
\quad
\subfigure[\sc Non-employment Rate]{\includegraphics[scale=0.4]{nonemprate_noptax.eps}\label{fig:noemprate_noptax}}
\end{figure}


\section{Conclusion}

\newpage

\bibliographystyle{CorpTax}
\bibliography{CorpTax}
\nocite{*}

\end{document}
